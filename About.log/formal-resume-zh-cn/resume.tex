% !TEX program = xelatex
% This is my resume
% Chinese translation
% by Kenji Mouri

\documentclass{resume}

\usepackage{lastpage}
\usepackage{fancyhdr}
\usepackage{linespacing_fix} % disable extra space before next section
\usepackage[fallback]{xeCJK}

\linespread{1.11} 

\begin{document}
\renewcommand\headrulewidth{0pt}

\name{卢起}

\basicInfo{
  \email{Mouri\_Naruto@Outlook.com} \textperiodcentered\
  \phone{182-6175-7560} \textperiodcentered\
  \github[MouriNaruto]{https://github.com/MouriNaruto} \textperiodcentered\
}

\section{教育经历}

\datedsubsection{\textbf{常熟理工学院}}{2016.09 -- 现在}
  汽车服务工程专业,已通过 CET-6,预计 2020 年 6 月毕业

\section{技能}
\begin{itemize}
  
  \item \textbf{程序语言}:
    熟悉 C/C++、C\#、Python
  
  \item \textbf{开发工具}:
    平常在 Windows 下使用 Visual Studio、IDA,有使用 Windows SDK (Win32、COM、ATL、WTL、WRL、C++/CX、C++/WinRT)、.NET (.NET Framework、.NET Core、WPF、Windows Forms、ASP.NET Core)、ROS (Kinetic、Melodic)、Qt、EDK II、Project Mu 框架进行应用开发的经验,同时也有使用 Git、VSTS 等团队协作工具的经验

\end{itemize}

\section{个人项目}

\datedsubsection{\textbf{NSudo}}{\url{https://github.com/M2Team/NSudo}}
一个系统管理工具,一般用于 TrustedInstaller 令牌获取。
\begin{itemize}
  \item 支持以 TrustedInstaller、System、当前用户、当前进程令牌运行应用
  \item 支持自定义快捷命令列表、多种命令行风格、多语言、高 DPI 缩放、UI 自动化、多处理器架构
  \item 提供包括特权、完整性、进程优先级、LUA 降权等细粒度的运行环境设置
  \item 精巧的体积、宽松的开源许可和提供 SDK 供用户进行二次开发
  \item 已经被 \href{https://forums.mydigitallife.net/threads/50572/}{MSMG ToolKit}、\href{https://forums.mydigitallife.net/threads/72203/}{Sledgehammer (原 WUMT Wrapper Script)}、\href{http://www.360.cn/n/10477.html}{"乱世"木马(链接为 360 分析报告)} 等项目采用
\end{itemize}

\datedsubsection{\textbf{Nagisa}}{\url{https://github.com/Project-Nagisa/Nagisa}}
一个基于 UWP 平台的文件传输工具,还处于早期开发阶段。
\begin{itemize}
  \item 提供对 HTTP、HTTPS、FTP、FTPS、WebSocket、WebSocket Secure 传输协议的支持
  \item 提供对多语言、后台下载、断点续传、单线程多任务下载的支持
  \item 为了推进本项目的开发,顺便改善了 OpenSSL 的\href{https://github.com/microsoft/openssl/pull/61}{微软分支的效能问题}和\href{https://github.com/openssl/openssl/blob/42b3f10b5e461496aab1f74d24103d6902ebfcd5/CHANGES#L350}{官方分支对 UWP 平台的支持}
\end{itemize}

\section{工作经历}

\datedsubsection{\textbf{初雨团队}}{2014.07 -- 现在}
\begin{itemize}
  \item 负责 Dism++ NCleaner 清理增强组件的开发,协助开发 Dism++ 映像热备份功能和多语言适配
  \item 创建了 \href{https://github.com/Chuyu-Team/MINT}{MINT} 项目,整合了 Windows 未文档化的用户模式 API 定义,且对 MSVC 工具链进行了优化
  \item 实现了 \href{https://github.com/Chuyu-Team/vc-ltl-samples}{VC-LTL 项目的使用示例},且协助开发 VC-LTL 的 Visual Studio 工具集成
  \item 学到了很多 C/C++ 编程、Windows 应用程序开发和 IDA 逆向相关的知识
\end{itemize}

\datedsubsection{\textbf{常熟理工学院车联网大数据实验室 (301实验室)}}{2018.03 -- 现在}
\begin{itemize}
  \item 协助开发实验室物联网平台,增加对西门子 S7-1200 PLC 的支持,且重构和扩充了部分实现
  \item 参与了自动驾驶避障算法的研究,并取得了一定成果
  \item 负责实验室自动驾驶测试车控制代码的第一次重构,并实现了远程驾驶的控制端支持
  \item 参与了第十六届“挑战杯”,并获取江苏省大学生课外学术科技作品竞赛二等奖
  \item 学到了很多 Python 编程、Linux 应用程序开发、ROS 平台和 ASP.NET Core 相关的知识
\end{itemize}

\section{其他}
\begin{itemize}

  \item 致力于编写使用最少的语法特性和第三方库的紧凑实现

  \item 开源贡献:推进和改善了 OpenSSL \href{https://github.com/openssl/openssl/pulls?q=is:pr+author:MouriNaruto+}{官方分支}和\href{https://github.com/microsoft/openssl/pulls?q=is:pr+author:MouriNaruto+}{微软分支}对 UWP 平台的支持、添加了\href{https://github.com/microsoft/winfile/pulls?q=is:pr+author:MouriNaruto+}{微软开源的旧版文件管理器 winfile} 的简体中文支持并共同确定了多语言支持目录结构、改善了 \href{https://github.com/covscript/covscript/pulls?q=is:pr+author:MouriNaruto+}{Covariant Script 编程语言}的 Windows 平台体验、创建了 \href{https://github.com/ffmpeginteropx/FFmpegInteropX/pulls?q=is:pr+author:MouriNaruto+}{FFmpegInteropX} 项目的 \href{https://github.com/ffmpeginteropx/FFmpegInteropX/tree/FFmpegUniversal}{FFmpeg 单动态链接库文件分支}
  
  \item 技术文章:\href{https://www.52pojie.cn/thread-512713-1-1.html}{《实现每显示器高 DPI 识别 (Per\-Monitor DPI Aware) 的注意事项》}、\href{https://www.52pojie.cn/thread-506556-1-1.html}{《开启 Win10 的文件资源管理器的每显示器 DPI 缩放 (Per\-Monitor DPI Aware) 支持》}、\href{http://bbs.pcbeta.com/viewthread-1567726-1-1.html}{《浅谈 Windows 10 Build 9879 的磁盘清理的 System Compression》}、\href{http://bbs.pcbeta.com/viewthread-1611980-1-1.html}{《浅谈 Metro App 的沙盒 AppContainer》}、\href{http://bbs.pcbeta.com/viewthread-1524688-1-1.html}{《自定义开始屏幕的大小》}、\href{http://bbs.pcbeta.com/viewthread-1535789-1-1.html}{《反汇编 Windows 系统还原代码的成果》}、\href{http://bbs.pcbeta.com/viewthread-1507617-1-1.html}{《Windows 系统还原新探 (Windows 系统还原的较深入研究)》}、\href{http://bbs.pcbeta.com/viewthread-1519551-1-1.html}{《原生集成 Windows 8/8.1 自带的 Windows Defender 病毒库的教程》}、\href{http://bbs.pcbeta.com/viewthread-1488892-1-1.html}{《使用系统自带工具把磁盘从MBR转换成 GPT 的教程》}
  
  \item 我很喜欢结识各种友人,与 \href{https://github.com/ice1000/}{ice1000}、\href{https://github.com/mikecovlee}{mikecovlee} 等进行过技术合作

\end{itemize}

\end{document}
